% Generated by Sphinx.
\def\sphinxdocclass{report}
\documentclass[letterpaper,10pt,french]{sphinxmanual}
\usepackage[utf8]{inputenc}
\DeclareUnicodeCharacter{00A0}{\nobreakspace}
\usepackage{cmap}
\usepackage[T1]{fontenc}
\usepackage[polish]{babel}
\usepackage{times}
\usepackage[Sonny]{fncychap}
\usepackage{longtable}
\usepackage{sphinx}
\usepackage{multirow}

\addto\captionsfrench{\renewcommand{\figurename}{Fig. }}
\addto\captionsfrench{\renewcommand{\tablename}{Tableau }}
\floatname{literal-block}{Code source }



\title{Plezuro}
\date{01 November 2015}
\release{1.0}
\author{Piotr Sroczkowski}
\newcommand{\sphinxlogo}{}
\renewcommand{\releasename}{Version}
\makeindex

\makeatletter
\def\PYG@reset{\let\PYG@it=\relax \let\PYG@bf=\relax%
    \let\PYG@ul=\relax \let\PYG@tc=\relax%
    \let\PYG@bc=\relax \let\PYG@ff=\relax}
\def\PYG@tok#1{\csname PYG@tok@#1\endcsname}
\def\PYG@toks#1+{\ifx\relax#1\empty\else%
    \PYG@tok{#1}\expandafter\PYG@toks\fi}
\def\PYG@do#1{\PYG@bc{\PYG@tc{\PYG@ul{%
    \PYG@it{\PYG@bf{\PYG@ff{#1}}}}}}}
\def\PYG#1#2{\PYG@reset\PYG@toks#1+\relax+\PYG@do{#2}}

\expandafter\def\csname PYG@tok@gd\endcsname{\def\PYG@tc##1{\textcolor[rgb]{0.63,0.00,0.00}{##1}}}
\expandafter\def\csname PYG@tok@gu\endcsname{\let\PYG@bf=\textbf\def\PYG@tc##1{\textcolor[rgb]{0.50,0.00,0.50}{##1}}}
\expandafter\def\csname PYG@tok@gt\endcsname{\def\PYG@tc##1{\textcolor[rgb]{0.00,0.27,0.87}{##1}}}
\expandafter\def\csname PYG@tok@gs\endcsname{\let\PYG@bf=\textbf}
\expandafter\def\csname PYG@tok@gr\endcsname{\def\PYG@tc##1{\textcolor[rgb]{1.00,0.00,0.00}{##1}}}
\expandafter\def\csname PYG@tok@cm\endcsname{\let\PYG@it=\textit\def\PYG@tc##1{\textcolor[rgb]{0.25,0.50,0.56}{##1}}}
\expandafter\def\csname PYG@tok@vg\endcsname{\def\PYG@tc##1{\textcolor[rgb]{0.73,0.38,0.84}{##1}}}
\expandafter\def\csname PYG@tok@m\endcsname{\def\PYG@tc##1{\textcolor[rgb]{0.13,0.50,0.31}{##1}}}
\expandafter\def\csname PYG@tok@mh\endcsname{\def\PYG@tc##1{\textcolor[rgb]{0.13,0.50,0.31}{##1}}}
\expandafter\def\csname PYG@tok@cs\endcsname{\def\PYG@tc##1{\textcolor[rgb]{0.25,0.50,0.56}{##1}}\def\PYG@bc##1{\setlength{\fboxsep}{0pt}\colorbox[rgb]{1.00,0.94,0.94}{\strut ##1}}}
\expandafter\def\csname PYG@tok@ge\endcsname{\let\PYG@it=\textit}
\expandafter\def\csname PYG@tok@vc\endcsname{\def\PYG@tc##1{\textcolor[rgb]{0.73,0.38,0.84}{##1}}}
\expandafter\def\csname PYG@tok@il\endcsname{\def\PYG@tc##1{\textcolor[rgb]{0.13,0.50,0.31}{##1}}}
\expandafter\def\csname PYG@tok@go\endcsname{\def\PYG@tc##1{\textcolor[rgb]{0.20,0.20,0.20}{##1}}}
\expandafter\def\csname PYG@tok@cp\endcsname{\def\PYG@tc##1{\textcolor[rgb]{0.00,0.44,0.13}{##1}}}
\expandafter\def\csname PYG@tok@gi\endcsname{\def\PYG@tc##1{\textcolor[rgb]{0.00,0.63,0.00}{##1}}}
\expandafter\def\csname PYG@tok@gh\endcsname{\let\PYG@bf=\textbf\def\PYG@tc##1{\textcolor[rgb]{0.00,0.00,0.50}{##1}}}
\expandafter\def\csname PYG@tok@ni\endcsname{\let\PYG@bf=\textbf\def\PYG@tc##1{\textcolor[rgb]{0.84,0.33,0.22}{##1}}}
\expandafter\def\csname PYG@tok@nl\endcsname{\let\PYG@bf=\textbf\def\PYG@tc##1{\textcolor[rgb]{0.00,0.13,0.44}{##1}}}
\expandafter\def\csname PYG@tok@nn\endcsname{\let\PYG@bf=\textbf\def\PYG@tc##1{\textcolor[rgb]{0.05,0.52,0.71}{##1}}}
\expandafter\def\csname PYG@tok@no\endcsname{\def\PYG@tc##1{\textcolor[rgb]{0.38,0.68,0.84}{##1}}}
\expandafter\def\csname PYG@tok@na\endcsname{\def\PYG@tc##1{\textcolor[rgb]{0.25,0.44,0.63}{##1}}}
\expandafter\def\csname PYG@tok@nb\endcsname{\def\PYG@tc##1{\textcolor[rgb]{0.00,0.44,0.13}{##1}}}
\expandafter\def\csname PYG@tok@nc\endcsname{\let\PYG@bf=\textbf\def\PYG@tc##1{\textcolor[rgb]{0.05,0.52,0.71}{##1}}}
\expandafter\def\csname PYG@tok@nd\endcsname{\let\PYG@bf=\textbf\def\PYG@tc##1{\textcolor[rgb]{0.33,0.33,0.33}{##1}}}
\expandafter\def\csname PYG@tok@ne\endcsname{\def\PYG@tc##1{\textcolor[rgb]{0.00,0.44,0.13}{##1}}}
\expandafter\def\csname PYG@tok@nf\endcsname{\def\PYG@tc##1{\textcolor[rgb]{0.02,0.16,0.49}{##1}}}
\expandafter\def\csname PYG@tok@si\endcsname{\let\PYG@it=\textit\def\PYG@tc##1{\textcolor[rgb]{0.44,0.63,0.82}{##1}}}
\expandafter\def\csname PYG@tok@s2\endcsname{\def\PYG@tc##1{\textcolor[rgb]{0.25,0.44,0.63}{##1}}}
\expandafter\def\csname PYG@tok@vi\endcsname{\def\PYG@tc##1{\textcolor[rgb]{0.73,0.38,0.84}{##1}}}
\expandafter\def\csname PYG@tok@nt\endcsname{\let\PYG@bf=\textbf\def\PYG@tc##1{\textcolor[rgb]{0.02,0.16,0.45}{##1}}}
\expandafter\def\csname PYG@tok@nv\endcsname{\def\PYG@tc##1{\textcolor[rgb]{0.73,0.38,0.84}{##1}}}
\expandafter\def\csname PYG@tok@s1\endcsname{\def\PYG@tc##1{\textcolor[rgb]{0.25,0.44,0.63}{##1}}}
\expandafter\def\csname PYG@tok@gp\endcsname{\let\PYG@bf=\textbf\def\PYG@tc##1{\textcolor[rgb]{0.78,0.36,0.04}{##1}}}
\expandafter\def\csname PYG@tok@sh\endcsname{\def\PYG@tc##1{\textcolor[rgb]{0.25,0.44,0.63}{##1}}}
\expandafter\def\csname PYG@tok@ow\endcsname{\let\PYG@bf=\textbf\def\PYG@tc##1{\textcolor[rgb]{0.00,0.44,0.13}{##1}}}
\expandafter\def\csname PYG@tok@sx\endcsname{\def\PYG@tc##1{\textcolor[rgb]{0.78,0.36,0.04}{##1}}}
\expandafter\def\csname PYG@tok@bp\endcsname{\def\PYG@tc##1{\textcolor[rgb]{0.00,0.44,0.13}{##1}}}
\expandafter\def\csname PYG@tok@c1\endcsname{\let\PYG@it=\textit\def\PYG@tc##1{\textcolor[rgb]{0.25,0.50,0.56}{##1}}}
\expandafter\def\csname PYG@tok@kc\endcsname{\let\PYG@bf=\textbf\def\PYG@tc##1{\textcolor[rgb]{0.00,0.44,0.13}{##1}}}
\expandafter\def\csname PYG@tok@c\endcsname{\let\PYG@it=\textit\def\PYG@tc##1{\textcolor[rgb]{0.25,0.50,0.56}{##1}}}
\expandafter\def\csname PYG@tok@mf\endcsname{\def\PYG@tc##1{\textcolor[rgb]{0.13,0.50,0.31}{##1}}}
\expandafter\def\csname PYG@tok@err\endcsname{\def\PYG@bc##1{\setlength{\fboxsep}{0pt}\fcolorbox[rgb]{1.00,0.00,0.00}{1,1,1}{\strut ##1}}}
\expandafter\def\csname PYG@tok@mb\endcsname{\def\PYG@tc##1{\textcolor[rgb]{0.13,0.50,0.31}{##1}}}
\expandafter\def\csname PYG@tok@ss\endcsname{\def\PYG@tc##1{\textcolor[rgb]{0.32,0.47,0.09}{##1}}}
\expandafter\def\csname PYG@tok@sr\endcsname{\def\PYG@tc##1{\textcolor[rgb]{0.14,0.33,0.53}{##1}}}
\expandafter\def\csname PYG@tok@mo\endcsname{\def\PYG@tc##1{\textcolor[rgb]{0.13,0.50,0.31}{##1}}}
\expandafter\def\csname PYG@tok@kd\endcsname{\let\PYG@bf=\textbf\def\PYG@tc##1{\textcolor[rgb]{0.00,0.44,0.13}{##1}}}
\expandafter\def\csname PYG@tok@mi\endcsname{\def\PYG@tc##1{\textcolor[rgb]{0.13,0.50,0.31}{##1}}}
\expandafter\def\csname PYG@tok@kn\endcsname{\let\PYG@bf=\textbf\def\PYG@tc##1{\textcolor[rgb]{0.00,0.44,0.13}{##1}}}
\expandafter\def\csname PYG@tok@o\endcsname{\def\PYG@tc##1{\textcolor[rgb]{0.40,0.40,0.40}{##1}}}
\expandafter\def\csname PYG@tok@kr\endcsname{\let\PYG@bf=\textbf\def\PYG@tc##1{\textcolor[rgb]{0.00,0.44,0.13}{##1}}}
\expandafter\def\csname PYG@tok@s\endcsname{\def\PYG@tc##1{\textcolor[rgb]{0.25,0.44,0.63}{##1}}}
\expandafter\def\csname PYG@tok@kp\endcsname{\def\PYG@tc##1{\textcolor[rgb]{0.00,0.44,0.13}{##1}}}
\expandafter\def\csname PYG@tok@w\endcsname{\def\PYG@tc##1{\textcolor[rgb]{0.73,0.73,0.73}{##1}}}
\expandafter\def\csname PYG@tok@kt\endcsname{\def\PYG@tc##1{\textcolor[rgb]{0.56,0.13,0.00}{##1}}}
\expandafter\def\csname PYG@tok@sc\endcsname{\def\PYG@tc##1{\textcolor[rgb]{0.25,0.44,0.63}{##1}}}
\expandafter\def\csname PYG@tok@sb\endcsname{\def\PYG@tc##1{\textcolor[rgb]{0.25,0.44,0.63}{##1}}}
\expandafter\def\csname PYG@tok@k\endcsname{\let\PYG@bf=\textbf\def\PYG@tc##1{\textcolor[rgb]{0.00,0.44,0.13}{##1}}}
\expandafter\def\csname PYG@tok@se\endcsname{\let\PYG@bf=\textbf\def\PYG@tc##1{\textcolor[rgb]{0.25,0.44,0.63}{##1}}}
\expandafter\def\csname PYG@tok@sd\endcsname{\let\PYG@it=\textit\def\PYG@tc##1{\textcolor[rgb]{0.25,0.44,0.63}{##1}}}

\def\PYGZbs{\char`\\}
\def\PYGZus{\char`\_}
\def\PYGZob{\char`\{}
\def\PYGZcb{\char`\}}
\def\PYGZca{\char`\^}
\def\PYGZam{\char`\&}
\def\PYGZlt{\char`\<}
\def\PYGZgt{\char`\>}
\def\PYGZsh{\char`\#}
\def\PYGZpc{\char`\%}
\def\PYGZdl{\char`\$}
\def\PYGZhy{\char`\-}
\def\PYGZsq{\char`\'}
\def\PYGZdq{\char`\"}
\def\PYGZti{\char`\~}
% for compatibility with earlier versions
\def\PYGZat{@}
\def\PYGZlb{[}
\def\PYGZrb{]}
\makeatother

\renewcommand\PYGZsq{\textquotesingle}

\begin{document}

\maketitle
\tableofcontents
\phantomsection\label{index::doc}


\includegraphics{logo.png}

Plezuro is a scripting language.

The main version is compiled to Javascript.

\begin{Verbatim}[commandchars=\\\{\}]
\PYG{n+nv+vg}{\PYGZdl{}f} \PYG{o}{=} \PYG{p}{\PYGZob{}}\PYG{n}{this} \PYG{o}{+} \PYG{n}{first} \PYG{o}{*} \PYG{l+m+mi}{2}\PYG{p}{\PYGZcb{}}\PYG{p}{;}
\PYG{n+nv+vg}{\PYGZdl{}y} \PYG{o}{=} \PYG{n}{f}\PYG{p}{(}\PYG{l+m+mi}{2}\PYG{p}{,} \PYG{n}{f}\PYG{p}{(}\PYG{l+m+mi}{5}\PYG{p}{,} \PYG{l+m+mi}{9}\PYG{p}{)}\PYG{p}{)}\PYG{p}{;}
\PYG{o}{[}\PYG{n}{y}\PYG{p}{,} \PYG{n}{y}\PYG{o}{]}
\end{Verbatim}


\chapter{Introduction}
\label{introduction/index:introduction}\label{introduction/index:welcome-to-plezuro-s-documentation}\label{introduction/index::doc}\label{introduction/index:id1}
This is introduction.

ghjkl
\begin{enumerate}
\item {} 
blabla
\begin{itemize}
\item {} 
ggggg

\item {} 
hhh
\begin{enumerate}
\item {} 
oo

\item {} 
kk

\end{enumerate}

\end{itemize}

\item {} 
dada
\begin{itemize}
\item {} 
rr

\end{itemize}

\item {} 
nothing

\end{enumerate}

\begin{tabulary}{\linewidth}{|L|L|L|}
\hline
\textsf{\relax 
A
} & \textsf{\relax 
B
} & \textsf{\relax 
C
}\\
\hline
hh
 & 
uu
 & 
oo
\\
\hline
kk
 & 
ll
 & 
mm
\\
\hline\end{tabulary}


\begin{center}LICENSE AGREEMENT
\end{center}

\section{Basics}
\label{introduction/basics/index:basics}\label{introduction/basics/index::doc}

\subsection{Conditions}
\label{introduction/basics/conditions/index:conditions}\label{introduction/basics/conditions/index::doc}
\begin{Verbatim}[commandchars=\\\{\}]
\PYG{n+nv+vg}{\PYGZdl{}x} \PYG{o}{=} \PYG{l+m+mi}{21}\PYG{p}{;}
\PYG{p}{\PYGZob{}}\PYG{n}{x} \PYG{o}{\PYGZgt{}} \PYG{l+m+mi}{0}\PYG{p}{\PYGZcb{}}\PYG{o}{.}\PYG{n}{if}\PYG{p}{(}\PYG{p}{\PYGZob{}}
  \PYG{n}{x}\PYG{o}{+}\PYG{o}{+}
\PYG{p}{\PYGZcb{}}\PYG{p}{)}\PYG{o}{.}\PYG{n}{elif}\PYG{p}{(}\PYG{p}{\PYGZob{}}\PYG{n}{x} \PYG{o}{\PYGZlt{}} \PYG{l+m+mi}{9}\PYG{p}{\PYGZcb{}}\PYG{p}{,} \PYG{p}{\PYGZob{}}
  \PYG{n}{x}\PYG{o}{\PYGZhy{}}\PYG{o}{\PYGZhy{}}
\PYG{p}{\PYGZcb{}}\PYG{p}{)}\PYG{o}{.}\PYG{n}{else}\PYG{p}{(}\PYG{p}{\PYGZob{}}
  \PYG{n}{x} \PYG{o}{*=} \PYG{l+m+mi}{2}
\PYG{p}{\PYGZcb{}}\PYG{p}{)}\PYG{p}{;}
\PYG{n}{x}
\end{Verbatim}


\subsection{Variables}
\label{introduction/basics/variables/index:variables}\label{introduction/basics/variables/index::doc}
Before we can use any variable, we have to declare it. It is pretty simple, just write the name of the variable and the dollar sign (`\$') immediately before it.
In next occurrences of a variable, you should write it without the dollar sign.

In Plezuro there is dynamic typing like in other dynamic languages such as Python, Ruby or Javascript. Also everything is a variable (including functions and modules).

\begin{Verbatim}[commandchars=\\\{\}]
\PYG{l+s+sr}{/}\PYG{l+s+sr}{/} \PYG{n}{number}
\PYG{n+nv+vg}{\PYGZdl{}a} \PYG{o}{=} \PYG{l+m+mi}{34}\PYG{p}{;}

\PYG{l+s+sr}{/}\PYG{l+s+sr}{/} \PYG{n}{string}
\PYG{n+nv+vg}{\PYGZdl{}b} \PYG{o}{=} \PYG{l+s+s1}{\PYGZsq{}abc\PYGZsq{}}\PYG{p}{;}

\PYG{l+s+sr}{/}\PYG{l+s+sr}{/} \PYG{n}{list}
\PYG{n+nv+vg}{\PYGZdl{}c} \PYG{o}{=} \PYG{o}{[}\PYG{n}{a}\PYG{p}{,} \PYG{n}{b}\PYG{o}{]}\PYG{p}{;}

\PYG{l+s+sr}{/}\PYG{l+s+sr}{/} \PYG{n}{associative} \PYG{n}{array}
\PYG{n+nv+vg}{\PYGZdl{}d} \PYG{o}{=} \PYG{l+s+sx}{\PYGZpc{}(}\PYG{l+s+sx}{\PYGZsq{}a\PYGZsq{}: a, \PYGZsq{}b\PYGZsq{}: b}\PYG{l+s+sx}{)}
\end{Verbatim}


\section{Origin}
\label{introduction/origin/index:origin}\label{introduction/origin/index::doc}
Where does Plezuro come from? Basically its author was not fully satisfied with any existing programming language.
So he thought about the inventing of a new one. The name comes from the Esperanto and means `pleasure'.


\section{Basic rules}
\label{introduction/rules/index::doc}\label{introduction/rules/index:basic-rules}
Plezuro is an imperative, object-oriented, functional, procedural and reflective programming language.

The main ideas are:
\begin{enumerate}
\item {} 
The code should be possibly short (as long as it is human readable).

\item {} 
Very simple syntax.

\item {} 
The code should be easy to read for a beginner programmer.

\item {} 
The power of the language should be based on the standard library, not its syntax.

\item {} 
Explicit is always better than implicit.

\item {} 
Everything is a variable.

\item {} 
Everything is an object.

\item {} 
Multiple inheritance.

\item {} 
There is no difference between a module, a class and a namespace.

\item {} 
No annotations and other additional syntax - everything is based on the basic syntax of the language.

\end{enumerate}



\renewcommand{\indexname}{Index}
\printindex
\end{document}
